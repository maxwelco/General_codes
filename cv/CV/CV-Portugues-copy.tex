\documentclass[11pt,]{article}
\usepackage[sc, osf]{mathpazo}
\usepackage{amssymb,amsmath}
\usepackage{ifxetex,ifluatex}
\usepackage{fixltx2e} % provides \textsubscript
\ifnum 0\ifxetex 1\fi\ifluatex 1\fi=0 % if pdftex
  \usepackage[T1]{fontenc}
  \usepackage[utf8]{inputenc}
\else % if luatex or xelatex
  \ifxetex
    \usepackage{mathspec}
  \else
    \usepackage{fontspec}
  \fi
  \defaultfontfeatures{Ligatures=TeX,Scale=MatchLowercase}
\fi
% use upquote if available, for straight quotes in verbatim environments
\IfFileExists{upquote.sty}{\usepackage{upquote}}{}
% use microtype if available
\IfFileExists{microtype.sty}{%
\usepackage{microtype}
\UseMicrotypeSet[protrusion]{basicmath} % disable protrusion for tt fonts
}{}
\usepackage[margin=1in]{geometry}




\setlength{\emergencystretch}{3em}  % prevent overfull lines
\providecommand{\tightlist}{%
  \setlength{\itemsep}{0pt}\setlength{\parskip}{0pt}}
\setcounter{secnumdepth}{0}
% Redefines (sub)paragraphs to behave more like sections
\ifx\paragraph\undefined\else
\let\oldparagraph\paragraph
\renewcommand{\paragraph}[1]{\oldparagraph{#1}\mbox{}}
\fi
\ifx\subparagraph\undefined\else
\let\oldsubparagraph\subparagraph
\renewcommand{\subparagraph}[1]{\oldsubparagraph{#1}\mbox{}}
\fi

% Now begins the stuff that I added.
% ----------------------------------

% Custom section fonts
\usepackage{sectsty}
\sectionfont{\rmfamily\mdseries\large\bf}
\subsectionfont{\rmfamily\mdseries\normalsize\itshape}


% Make lists without bullets
\renewenvironment{itemize}{
  \begin{list}{}{
    \setlength{\leftmargin}{1.5em}
  }
}{
  \end{list}
}


% Make parskips rather than indent with lists.
\usepackage{parskip}
\usepackage{titlesec}
\titlespacing\section{0pt}{12pt plus 4pt minus 2pt}{4pt plus 2pt minus 2pt}
\titlespacing\subsection{0pt}{12pt plus 4pt minus 2pt}{4pt plus 2pt minus 2pt}

% Use fontawesome. Note: you'll need TeXLive 2015. Update.
\usepackage{fontawesome}

% Fancyhdr, as I tend to do with these personal documents.
\usepackage{fancyhdr,lastpage}
\pagestyle{fancy}
\renewcommand{\headrulewidth}{0.0pt}
\renewcommand{\footrulewidth}{0.0pt}
\lhead{}
\chead{}
\rhead{}
\lfoot{
\cfoot{\scriptsize  Maxwel Coura Oliveira, PhD - Curriculum Vitae }}
\rfoot{\scriptsize \thepage/{\hypersetup{linkcolor=black}\pageref{LastPage}}}

% Always load hyperref last.
\usepackage{hyperref}
\PassOptionsToPackage{usenames,dvipsnames}{color} % color is loaded by hyperref

\hypersetup{unicode=true,
            pdftitle={Maxwel Coura Oliveira, PhD:  Curriculum Vitae (Curriculum Vitae)},
            pdfauthor={Maxwel Coura Oliveira, PhD},
            pdfkeywords={R Markdown, academic CV, template},
            colorlinks=true,
            linkcolor=blue,
            citecolor=Blue,
            urlcolor=blue,
            breaklinks=true, bookmarks=true}
\urlstyle{same}  % don't use monospace font for urls

\begin{document}


\centerline{\huge \bf Maxwel Coura Oliveira, PhD}

\vspace{2 mm}

\hrule

\vspace{2 mm}

\moveleft.5\hoffset\centerline{Professor, Universidade do Oeste Paulista e Pesquisador, Universidade de
Wisconsin-Madison}
\moveleft.5\hoffset\centerline{Rodovia Raposo Tavares, km 572, · Departamento de Agronomia ·
Pres.~Prudente · SP, BRA 9067-175}
\moveleft.5\hoffset\centerline{ \faEnvelopeO \hspace{1 mm} \href{mailto:}{\tt \href{mailto:max.oliveira@wisc.edu}{\nolinkurl{max.oliveira@wisc.edu}}} \hspace{1 mm}  \faPhone \hspace{1 mm}  +55 (18) 99690-1070  \hspace{1 mm}  \faGithub \hspace{1 mm} \href{http://github.com/maxwelco}{\tt maxwelco} \hspace{1 mm}   \faTwitter \hspace{1 mm} \href{https:/twitter.com/maxwelco}{\tt maxwelco} \hspace{1 mm}  \faGlobe \hspace{1 mm} \href{http://maxweeds.rbind.io}{\tt maxweeds.rbind.io}   }

\vspace{2 mm}

\hrule


\hypertarget{educauxe7uxe3o}{%
\section{EDUCAÇÃO}\label{educauxe7uxe3o}}

• \emph{Universidade de Nebraska-Lincoln}, Ph.D.~Agronomia (Planta
daninha) \hfill Dez 2017

• \emph{Universidade Federal dos Vales do Jequitinhonha e Mucuri}, MSc.
Produção vegetal (Planta daninha) \hfill Jul 2013

• \emph{Universidade Federal dos Vales do Jequitinhonha e Mucuri}, BSc.
Agronomia \hfill Jul 2011

\hypertarget{experiuxeancia-profissional-selecionadas}{%
\section{EXPERIÊNCIA PROFISSIONAL
SELECIONADAS}\label{experiuxeancia-profissional-selecionadas}}

• \emph{Professor}, Universidade do Oeste Paulista · Presidente Prudente
· SP · BRA \hfil Fev 2020 · Atual\\
\emph{Objetivo}: Professor e pesquisador de plantas daninhas. Ministra
aulas na graduação e pós-graduação.

• \emph{Pesquisador Associado},
\href{http://www.wiscweeds.info/}{WiscWeeds Lab} · Universidade de
Wisconsin-Madison · Madison · WI · EUA \hfil Jan 2018 · Atual\\
\emph{Objetivo}: realizar pesquisa inovadora e colaborativa e de
extensão em biologia, ecologia e manejo de ervas daninhas problemáticas
em milho, soja e cânhamo sob a supervisão do Dr.~Rodrigo Werle

• \emph{Doutoramento}, \href{https://agronomy.unl.edu/knezevic}{Knezevic
Lab} Universidade de Nebraska-Lincoln, Lincoln · NE · EUA \hfil Jan,
2014 até Dez 2017\\
\emph{Objetivo}: Pesquisa de doutorado em Sistemas de Produção, Biologia
e Ecologia de Plantas Daninhas, Manejo Integrado de Plantas Daninhas,
Evolução da Resistência a Herbicidas e ensaios de eficácia de herbicidas
sob a supervisão do Dr.~Stevan Knezevic

• \emph{Trainee}, \href{http://www.timacagro.com.br/}{TIMAC Agro} ·
Paragominas · PA · Brasil \hfill Jan a Jun 2013\\
\emph{Objetivo}: Treinado para ser um supervisor de equipe de vendas de
fertilizantes, aquisição de talentos e marketing

• \emph{Mestrado}, \href{http://www.ufvjm.edu.br/}{Universidade Federal
dos Vales do Jequitinhonha e Mucuri} · Diamantina · MG · Brasil
\hfill Jul 2011 a Dez 2012\\
\emph{Objetivo}: Mestrado em biologia, competição e manejo integrado de
ervas daninhas sob supervisão do Dr.~José Barbosa dos Santos

• \emph{Estágio internacional}, \href{https://top.osu.edu/}{The Ohio
Program} · OH · EUA \hfill Abr 2010 a Mar 2011\\
\emph{Objetivo}: Mão de fazenda em uma fazenda de vegetais em Michigan
(EUA) e assistente em uma estufa de tomate hidropônico em Nova York, EUA

\hypertarget{aulas-ministradas}{%
\section{AULAS MINISTRADAS}\label{aulas-ministradas}}

• \emph{Universidade do Oeste Paulista}, Manejo de Plantas Daninhas
\hfill  2020

• \emph{Universidade do Oeste Paulista}, Análise estatística no R
\hfill 2019/2020

• \emph{Universidade de Wisconsin-Madison}, Peparação para o Camp de
Herbologia \hfill 2018

• \emph{Universidade de Nebraska-Lincoln}, Lab de Plantas Invasoras
\hfill 2016/2017

\hypertarget{monitorias}{%
\section{MONITORIAS}\label{monitorias}}

• \emph{Universidade Federal dos Vales do Jequitinhonha e Mucuri}, Lab
de Plantas Daninhas \hfill 2011

• \emph{Universidade Federal dos Vales do Jequitinhonha e Mucuri}, Lab
de Melhoramento Genético \hfill 2009

\hypertarget{pesquisador-visitante}{%
\section{PESQUISADOR VISITANTE}\label{pesquisador-visitante}}

• \emph{Universidade de Illinois Urbana-Champaign} · IL · EUA
\hfill Setembro 2017\\
Pesquisa sobre mecanísmo de resistência de \emph{Amaranthus palmeri}
para herbicidas inibidores da PPO sob supervisão de Dr.~Pat Tranel

• \emph{Universidade do Estado do Colorado--Fort Collins} · CO · EUA
\hfill Maio 2017\\
Pesquisa no lab sobre resistencia de \emph{Amaranthus tuberculatus} aos
herbicidas inibidores da HPPD sob supervisão de Dr.~Todd Gaines

• \emph{Universidade do Estado do Colorado--Fort Collins} · CO · EUA
\hfill Setembro 2016\\
Pesquisa no lab sobre fisiologia dos hebricidas inibidores da HPPD em
\emph{Amaranthus tuberculatus} sob supervisão de Dr.~Franck Dayan

• \emph{Universidade Federal de Viçosa-Viçosa} · MG · Brasil
\hfill Agosto 2012 à Dezembro 2012\\
Pesquisa no campo, lab e estufas com comportamento de herbicidas no solo
e manejo de plantas daninhas sob supervisão do Dr.~Antonio Alberto Silva

\hypertarget{honras-acaduxeamicas-e-pruxeamios}{%
\section{HONRAS ACADÊMICAS E
PRÊMIOS}\label{honras-acaduxeamicas-e-pruxeamios}}

• \emph{Congresso da North Central Weed Science Society}, Saint louis ·
MO · EUA \hfill Dez 2017\\
Melhor Apresentação de Poster (Resistência de Plantas Daninhas)

• \emph{Competição da North Central Weed Science Society}, Ames · IA ·
EUA \hfill Jul 2017\\
Primeiro Colocado Time e Primeiro Colocado Individual em Identificação
de Herbicidas

• \emph{Universidade de Nebraska-Lincoln}, Lincoln · NE · EUA \hfill Mar
2017\\
Travel Grant (\$400) to attend the 2017 Global Herbicide Resistant
Conference

• \emph{Campeonato de Herbologia da North Central Weed Science Society},
West Laffayete · EUA \hfill Jul 2016\\
Terceiro Colocado Time

• \emph{Congresso da Weed Science Society of America}, San Juan · PR
\hfill Fev 2015\\
Primeiro Colocado Poster (Biologia de Plantas Daninhas)

• \emph{Campeonato de Herbologia da Western Weed Science Society},
Columbus · OH · EUA \hfill Julho 2015\\
Terceiro Colocado Time e Terceiro Colocado individual

\hypertarget{membro-de-sociedades-profissionais}{%
\section{MEMBRO DE SOCIEDADES
PROFISSIONAIS}\label{membro-de-sociedades-profissionais}}

• Weed Science Society of America

• North Central Weed Science Society

\hypertarget{manuscritos-em-preparauxe7uxe3o}{%
\section{MANUSCRITOS EM
PREPARAÇÃO}\label{manuscritos-em-preparauxe7uxe3o}}

• \textbf{Oliveira MC}, Butts T, Conley S, e Werle R. Tolerance of US
Midwest Soybean Cultivars to Preemergent Applications of Metribuzin e
Sulfentrazone. Enviar para: \emph{Weed Technology}

• Smith D, \textbf{Oliveira MC}, Davis V, e Werle R. (2019) Herbicide
carryover evaluation following commonly applied silage corn e soybean
herbicides. Enviar para: \emph{Agronomy}

• \textbf{Oliveira MC}, da Silva AL, Ulguim AR, Werle R. Assessment of
Weed Management Strategies e Challenges in Brasilian Cropping Systems.
Enviar para: \emph{Weed Technology}

\hypertarget{manuscritos-submetidos}{%
\section{MANUSCRITOS SUBMETIDOS}\label{manuscritos-submetidos}}

• \textbf{Oliveira MC}, Giacomini D, Tranel P, e Werle R. Molecular e
greenhouse validation of field evolved resistance to glyphosate e
PPO-inhibitors in Nebraska. Submission: \emph{Pest Management Science}

• \textbf{Oliveira MC}, Osipitan OA, Begcy K, e Werle R. Cover crops,
hormones e herbicides: priming an Integrated Weed Management Strategy.
Enviado para: \emph{Plant Science} (convidado)

\hypertarget{publicauxe7uxf5es-selecionadas}{%
\section{PUBLICAÇÕES
SELECIONADAS}\label{publicauxe7uxf5es-selecionadas}}

• Soltani S, \textbf{Oliveira MC}, Alves GS, Werle R, Norsworthy JK,
Sprague CL, Young BG, Reynolds DB, Brown A, e Sikkema PH (2019)
Off-Target Movement Assessment of Dicamba in North America. in press
\emph{Weed Technology}. doi:
\href{https://doi.org/10.1017/wet.2020.17}{10.1017/wet.2020.17}

• \textbf{Oliveira MC}, Butts L, e Werle R (2019) Survey of cover crop
management in Nebraska. 9:1-14 \emph{Agriculture}, convidado: ``Cover
Crops''. doi:
\href{https://doi.org/10.3390/agriculture9060124}{10.3390/agriculture9060124}

• Knezevic SZ, Pavlovic P, Barnes ER, Beiermann C, \textbf{Oliveira MC},
Lawrence N, Scott JE, e Jhala AJ (2019) Critical Time for Weed Removal
in Glyphosate-Resistant Soybean as Influenced by Preemergence
Herbicides. 33-393:399. \emph{Weed Technology}. doi:
\href{https://doi.org/10.1017/wet.2019.18}{10.1017/wet.2019.18}

• Ferreira EA, Paiva MCG, Pereira GAM, \textbf{Oliveira MC}, Silva, EB
(2019) Fitossociologia de plantas daninhas na cultura do milho submetida
à aplicação de doses de nitrogênio, \emph{Journal of Neotropical
Agriculture}, 6:100-107 (in Portuguese with English abstract). doi:
\href{https://doi.org/10.32404/rean.v6i2.2710}{10.32404/rean.v6i2.2710}

• Werle R, \textbf{Oliveira MC}, Jhala AJ, Klein R, Proctor CA, e Rees J
(2018) Survey of Nebraska farmers' adoption of dicamba-tolerant soybean
technology e off-target movement. \emph{Weed Technology}, 32:754--761.
doi: \href{https://doi.org/10.1017/wet.2018.62}{10.1017/wet.2018.62}

• \textbf{Oliveira MC}, Gaines TA, Patterson EP, Jhala AJ, Irmak S,
Amundsen K, e Knezevic SZ (2018) Interspecific e intraspecific transfer
of metabolism based mesotrione resistance transfer in dioecious weedy
\emph{Amaranthus}. \emph{Plant Journal}. 96:1051--1063. doi:
\href{https://onlinelibrary.wiley.com/doi/full/10.1111/tpj.14089}{10.1111/tpj.14089}

• Knezevic SZ, Ospitan AO, \textbf{Oliveira MC}, e Scott JE (2018)
\emph{Lythrum salicaria} (purple loosestrife) control with herbicides:
multiyear applications. \emph{Invasive Plant Science e Management}.
11:143--154. doi:
\href{https://doi.org/10.1017/inp.2018.17}{10.1017/inp.2018.17}

• \textbf{Oliveira MC}, Pereira GAM, Ferreira EA, Barbosa JB, Knezevic
SZ, e Werle R (2018) Additive design: the concept e data analysis.
\emph{Weed Research}, 58:338--347. doi:
\href{https://onlinelibrary.wiley.com/doi/full/10.1111/wre.12317}{10.1111/wre.12317}

• \textbf{Oliveira MC}, Gaines TA, Jhala AJ, e Knezevic SZ (2018)
Inheritance of mesotrione resistance in an (\emph{Amaranthus
tuberculatus}) (var. \emph{rudis}) population from Nebraska, EUA.
\emph{Frontiers in Plant Science}, 9:60. doi:
\href{https://www.frontiersin.org/articles/10.3389/fpls.2018.00060/full}{10.3389/fpls.2018.00060}

• Miller JJ, Schepers JS, Shapiro CA, Arnenson NJ, Eskridge KM,
\textbf{Oliveira MC}, e Giesler LJ (2018) Characterizing soybean vigor e
productivity using multiple crop canopy sensor readings. \emph{Field
Crops Research}, 216:22--31. doi:
\href{https://www.sciencedirect.com/science/article/pii/S0378429017310341}{10.1016/j.fcr.2017.11.006}

• \textbf{Oliveira MC}, Gaines TA, Dayan FE, Patterson EL, Jhala AJ, e
Knezevic SZ (2018) Reversing resistance to tembotrione in an
\emph{Amaranthus tuberculatus} (syn. \emph{rudis}) population from
Nebraska, EUA with cytochrome P450 inhibitors. \emph{Pest Management
Science}, 74:2296--2305. (Publicação convidado) doi:
\href{https://onlinelibrary.wiley.com/doi/full/10.1002/ps.4697}{10.1002/ps.4697}

• \textbf{Oliveira MC}, Jhala A, Gaines T, Irmak S, Amundsen K, Scott
JE, e Knezevic SZ (2017) Confirmation e Control of HPPD-inhibiting
Herbicide-Resistant Waterhemp (\emph{Amaranthus tuberculatus}) in
Nebraska. \emph{Weed Technology}, 31:67--79. doi:
\href{https://www.cambridge.org/core/journals/weed-technology/article/confirmation-e-control-of-hppdinhibiting-herbicideresistant-waterhemp-amaranthus-tuberculatus-in-nebraska/69C31C7039DBE3FD49A55C73EEE5F2EE}{10.1017/wet.2016.4}

• Braga RR, Santos JB, Zanuncio JC, Bibiano CS, Ferreira EA,
\textbf{Oliveira MC}, Silva DV, e Serrāo JE (2016) Effect of growing
\emph{Brachiaria brizantha} on phytoremediation of picloram under
different pH environments. \emph{Ecological Engineering}, 94:102--106.
doi:
\href{https://www.sciencedirect.com/science/article/pii/S0925857416302853}{10.1016/j.ecoleng.2016.05.050}

\hypertarget{palestras-e-treinamentos}{%
\section{PALESTRAS E TREINAMENTOS}\label{palestras-e-treinamentos}}

• \emph{Valoriza Agronegocios}, Patos de Minas · MG · Brasil \hfill Jan
2019\\
Título: ``Treinamento resistência, dicamba, identificação de plantas
daninhas e herbicidas''

• \emph{II IMAST}, Dracena · SP · Brasil \hfill Setembro 2018\\
Título: ``Da resistência ao dicamba: o desafio no manejo de plantas
daninhas nos EUA''

• \emph{JF Crow Institute for the Study of Evolution}, Madison · WI ·
EUA \hfill Set 2018\\
Título: Evolução de \emph{Amaranthus tuberculatus} aos herbicidas
inibidores da 4-hydroxyphepylpiruvate dioxygenase

\hypertarget{palestras-em-eventos-de-extensuxe3o}{%
\section{PALESTRAS EM EVENTOS DE
EXTENSÃO}\label{palestras-em-eventos-de-extensuxe3o}}

• Evolução de Resistência em \emph{Amaranthus tuberculatus}.\\
Waterhemp worshop, Chippewa Falls, Bangor, Janesville, Appleton, WI,
EUA, Mar 2019

• Dia de Campo Planta Daninhas, Arlington e Lancaster · WI · EUA
\hfill Jul 2018

• Dia de Campo da Mandioca: Especificidades e Práticas, Diamantina, MG,
Brasil \hfill Out 2011

\hypertarget{resumos-de-congressos}{%
\section{RESUMOS DE CONGRESSOS}\label{resumos-de-congressos}}

• \emph{70+ resumos em congressos no Brasil e EUA entre 2008 e 2020}

\hypertarget{projetos-aprovados}{%
\section{PROJETOS APROVADOS}\label{projetos-aprovados}}

• \href{https://oeo.cals.wisc.edu}{UW-Madison Division of Extension e
UW-Madison CALS} \hfill Mar 2019 Título: Adaptation of Industrial Hemp
(\emph{Cannabis sativa}) to Agronomic practices in Wisconsin Pago:
US\$35,000

• \href{http://wicorn.org}{Wisconsin Corn Promotion Board} \hfill Jan
2019\\
Título: Challenging the Weed Management Challenges in Wisconsin Corn
Production Pago: US\$30,000

• \href{http://www.capes.gov.br/}{CAPES} Foundation \hfill September
2013\\
Título: Evolution of HPPD-resistant waterhemp (\emph{Amaranthus
tuberculatus}) in Nebraska\\
Pago: US\$130,000 - Doutorado na Universidade de Nebraska-Lincoln

\hypertarget{manuais}{%
\section{MANUAIS}\label{manuais}}

• Residual Control of Waterhemp with PRE-emergence Herbicides in Soybean
\hfill Fev 2020
\href{http://www.wiscweeds.info/img/2018\%202019\%20waterhemp\%20challenge/PreEmergence_waterhempFINAL.pdf}{Link}

• Cover Crops 101 \hfill Ago 2019\\
\href{https://learningstore.extension.wisc.edu/collections/farming/products/cover-crops-101}{Link}

• Chave para Identificação de Injúrias Causadas por Herbicidas
\hfill Set 2018\\
\href{http://www.wiscweeds.info/img/2018\%20Herbicide\%20SOA\%20injury\%20chart/2018_HerbicideInjury_EN.pdf}{Inglês}
e
\href{https://maxweeds.rbind.io/files/Herbicide\%20chart/2018_HerbicideInjury_BRA.pdf}{Português}.

• UW Waterhemp Challenge: Preliminary Report Comparison of Soil Residual
Herbicides \hfill Ago 2018\\
\href{http://www.wiscweeds.info/img/2018\%20waterhemp\%20challenge/Waterhemp\%20Management\%20PRE\%20Comparison_Lancaster\%20WI\%202018.pdf}{Link}

\hypertarget{programauxe7uxe3o}{%
\section{PROGRAMAÇÃO}\label{programauxe7uxe3o}}

• Terminal

• R

• C++

• GitHub

• Agricultural Research Management (ARM)

• SAS

\hypertarget{projetos-pessoais}{%
\section{PROJETOS PESSOAIS}\label{projetos-pessoais}}

• Open Source Weed Science -- Um repositório com tutoriais de análise e
visualização de dados publicados em artigos de planta daninha e áreas
relacionadas: \url{https://openweedsci.org}

• Meu blog pessoal - Blog: \url{https://maxweeds.rbind.io}

\hypertarget{linguas}{%
\section{LINGUAS}\label{linguas}}

• Português (Nativo)

• Inglês (Fluente)

• Espanhol (Intermediário)

\end{document}