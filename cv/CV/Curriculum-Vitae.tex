\documentclass[11pt,]{article}
\usepackage[sc, osf]{mathpazo}
\usepackage{amssymb,amsmath}
\usepackage{ifxetex,ifluatex}
\usepackage{fixltx2e} % provides \textsubscript
\ifnum 0\ifxetex 1\fi\ifluatex 1\fi=0 % if pdftex
  \usepackage[T1]{fontenc}
  \usepackage[utf8]{inputenc}
\else % if luatex or xelatex
  \ifxetex
    \usepackage{mathspec}
  \else
    \usepackage{fontspec}
  \fi
  \defaultfontfeatures{Ligatures=TeX,Scale=MatchLowercase}
\fi
% use upquote if available, for straight quotes in verbatim environments
\IfFileExists{upquote.sty}{\usepackage{upquote}}{}
% use microtype if available
\IfFileExists{microtype.sty}{%
\usepackage{microtype}
\UseMicrotypeSet[protrusion]{basicmath} % disable protrusion for tt fonts
}{}
\usepackage[margin=1in]{geometry}




\setlength{\emergencystretch}{3em}  % prevent overfull lines
\providecommand{\tightlist}{%
  \setlength{\itemsep}{0pt}\setlength{\parskip}{0pt}}
\setcounter{secnumdepth}{0}
% Redefines (sub)paragraphs to behave more like sections
\ifx\paragraph\undefined\else
\let\oldparagraph\paragraph
\renewcommand{\paragraph}[1]{\oldparagraph{#1}\mbox{}}
\fi
\ifx\subparagraph\undefined\else
\let\oldsubparagraph\subparagraph
\renewcommand{\subparagraph}[1]{\oldsubparagraph{#1}\mbox{}}
\fi

% Now begins the stuff that I added.
% ----------------------------------

% Custom section fonts
\usepackage{sectsty}
\sectionfont{\rmfamily\mdseries\large\bf}
\subsectionfont{\rmfamily\mdseries\normalsize\itshape}


% Make lists without bullets
\renewenvironment{itemize}{
  \begin{list}{}{
    \setlength{\leftmargin}{1.5em}
  }
}{
  \end{list}
}


% Make parskips rather than indent with lists.
\usepackage{parskip}
\usepackage{titlesec}
\titlespacing\section{0pt}{12pt plus 4pt minus 2pt}{4pt plus 2pt minus 2pt}
\titlespacing\subsection{0pt}{12pt plus 4pt minus 2pt}{4pt plus 2pt minus 2pt}

% Use fontawesome. Note: you'll need TeXLive 2015. Update.
\usepackage{fontawesome}

% Fancyhdr, as I tend to do with these personal documents.
\usepackage{fancyhdr,lastpage}
\pagestyle{fancy}
\renewcommand{\headrulewidth}{0.0pt}
\renewcommand{\footrulewidth}{0.0pt}
\lhead{}
\chead{}
\rhead{}
\lfoot{
\cfoot{\scriptsize  Maxwel Coura Oliveira, PhD - Curriculum Vitae }}
\rfoot{\scriptsize \thepage/{\hypersetup{linkcolor=black}\pageref{LastPage}}}

% Always load hyperref last.
\usepackage{hyperref}
\PassOptionsToPackage{usenames,dvipsnames}{color} % color is loaded by hyperref

\hypersetup{unicode=true,
            pdftitle={Maxwel Coura Oliveira, PhD:  Curriculum Vitae (Curriculum Vitae)},
            pdfauthor={Maxwel Coura Oliveira, PhD},
            pdfkeywords={R Markdown, academic CV, template},
            colorlinks=true,
            linkcolor=blue,
            citecolor=Blue,
            urlcolor=blue,
            breaklinks=true, bookmarks=true}
\urlstyle{same}  % don't use monospace font for urls

\begin{document}


\centerline{\huge \bf Maxwel Coura Oliveira, PhD}

\vspace{2 mm}

\hrule

\vspace{2 mm}

\moveleft.5\hoffset\centerline{Postdoctoral Research Associate, University of Wisconsin-Madison}
\moveleft.5\hoffset\centerline{1575 Linden Drive · Department of Agronomy · Madison, Wisconsin, USA
53706}
\moveleft.5\hoffset\centerline{ \faEnvelopeO \hspace{1 mm} \href{mailto:}{\tt \href{mailto:max.oliveira@wisc.edu}{\nolinkurl{max.oliveira@wisc.edu}}} \hspace{1 mm}  \faPhone \hspace{1 mm}  +1 402 580 6652  \hspace{1 mm}  \faGithub \hspace{1 mm} \href{http://github.com/maxwelco}{\tt maxwelco} \hspace{1 mm}   \faTwitter \hspace{1 mm} \href{https:/twitter.com/maxwelco}{\tt maxwelco} \hspace{1 mm}  \faGlobe \hspace{1 mm} \href{http://maxweeds.rbind.io}{\tt maxweeds.rbind.io}   }

\vspace{2 mm}

\hrule


\hypertarget{education}{%
\section{EDUCATION}\label{education}}

• \emph{University of Nebraska-Lincoln}, Ph.D.~Agronomy (Weed Science)
\hfill December 2017

• \emph{Federal University of Jequitinhonha and Mucuri Valleys}, MSc.
Weed Science \hfill July 2013

• \emph{Federal University of Jequitinhonha and Mucuri Valleys}, BSc.
Agronomy \hfill July 2011

\hypertarget{professional-experience}{%
\section{PROFESSIONAL EXPERIENCE}\label{professional-experience}}

• \emph{Professor}, \href{http://www.unoeste.br/}{University of Western
Sao Paulo} · Presidente Prudente · SP · Brazil February 2020\\
\emph{Goal}: conduct innovative and collaborative research and teaching
on biology, ecology and management of troublesome weeds in Western Sao
Paulo.

• \emph{Postdoctoral Research Associate},
\href{http://www.wiscweeds.info/}{WiscWeeds Lab} · University of
Wisconsin-Madison · Madison · WI · USA January 2018\\
\emph{Goal}: conduct innovative and collaborative research and extension
programming on biology, ecology and management of troublesome weeds in
corn, soybeans and industrial hemp under Dr.~Rodrigo Werle's supervision

• \emph{Graduate Research Assistant},
\href{https://agronomy.unl.edu/knezevic}{Knezevic Lab} University of
Nebraska-Lincoln, Lincoln · NE · USA \hfil January, 2014 to December
2017\\
\emph{Goal}: Doctoral research in Weed Biology and Ecology, Integrated
Weed Management, Herbicide Resistance Evolution and Management, and
Herbicide efficacy trials under Dr.~Stevan Knezevic's supervision

• \emph{Sales Management Trainee},
\href{http://www.timacagro.com.br/}{TIMAC Agro} · Paragominas · PA ·
Brazil \hfill January to June 2013\\
\emph{Goal}: Trained to be a supervisor of fertilizers sales team,
talent acquisition, and marketing

• \emph{Graduate Research Assistant},
\href{http://www.ufvjm.edu.br/}{Federal University of Jequitinhonha and
Mucuri Valleys} · Diamantina · MG · Brazil \hfill July 2011 to December
2012\\
\emph{Goal}: Masters research in weed Biology, competition, and
integrated weed management under Dr.~José Barbosa dos Santos'
supervision

• \emph{Undergraduate Senior Trainee}, \href{https://top.osu.edu/}{The
Ohio Program} · OH · USA \hfill April 2010 to March 2011\\
\emph{Goal}: farm hand in a vegetable farm in Michigan (USA) and
greenhouse assistant in a hydroponic tomato greenhouse in New York, USA.

• \emph{Undergraduate Junior Trainee}, Pulveriza Fertilizers · Matipó ·
MG · Brazil \hfill July 2009\\
\emph{Goal}: work in a industry of foliar fertilizers for corn,
vegetables and coffee trees

• \emph{Coordinator}, \href{https://junior.creadf.org.br/}{CREA Jr} ·
Diamantina · MG · Brazil \hfill January to December 2009\\
\emph{Goal}: leading the club for mentoring students, connecting
students to professional engineers, organizing seminars, prospective of
new members

• \emph{Undergraduate Intern},
\href{http://www.emater.mg.gov.br/}{EMATER MG} · Santana do Paraíso · MG
· Brazil \hfill July 2008 \emph{Goal}: extension work with small
vegetables growers

\newpage

\hypertarget{teaching-experience}{%
\section{TEACHING EXPERIENCE}\label{teaching-experience}}

• \emph{University of Wisconsin-Madison}, Weeds Contest Prep Course
\hfill Spring 2018

• \emph{University of Nebraska-Lincoln}, Invasive Plants Lab
\hfill Spring 2016/2017

• \emph{Federal University of Jequitinhonha and Mucuri Valleys}, Weed
Manag Lab \hfill Spring 2011

• \emph{Federal University of Jequitinhonha and Mucuri Valleys}, Crop
Improv Lab \hfill Spring/Fall 2009

\hypertarget{visitor-scholar}{%
\section{VISITOR SCHOLAR}\label{visitor-scholar}}

• \emph{University of Illinois Urbana-Champaign} · IL · USA
\hfill September 2017\\
Laboratory research on PPO resistance mechanism in \emph{Amaranthus
palmeri} from Nebraska under Dr.~Pat Tranel's supervision

• \emph{Colorado State University--Fort Collins} · CO · USA \hfill May
2017\\
Laboratory research on HPPD resistance mechanism in \emph{Amaranthus
tuberculatus} from Nebraska under Dr.~Todd Gaines' supervision

• \emph{Colorado State University--Fort Collins} · CO · USA
\hfill September 2016\\
Laboratory research on herbicide physiology in \emph{Amaranthus
tuberculatus} from Nebraska under Dr.~Franck Dayan's supervision

• \emph{Federal University of Viçosa-Viçosa} · MG · Brazil \hfill August
2012 to December 2012\\
Field, laboratory, and greenhouse research with fate of herbicide in the
soil, and integrated weed management under Dr.~Antonio Alberto Silva's
supervision

\hypertarget{academic-honors-and-awards}{%
\section{ACADEMIC HONORS AND AWARDS}\label{academic-honors-and-awards}}

• \emph{North Central Weed Science Society Meeting}, Saint louis · MO ·
USA \hfill December 2017\\
1\textsuperscript{st} Place Outstanding Poster Presentation (Herbicide
Resistance section)

• \emph{North Central Weed Science Society Weed Contest}, Ames · IA ·
USA \hfill July 2017\\
1\textsuperscript{st} Place Graduate Team and 1\textsuperscript{st}
Place Overall Individual Herbicide ID

• \emph{University of Nebraska-Lincoln}, Lincoln · NE · USA \hfill March
2017\\
Travel Grant (\$400) to attend the 2017 Global Herbicide Resistant
Conference

• \emph{North Central Weed Science Society Weed Contest}, West Laffayete
· USA \hfill July 2016\\
3\textsuperscript{rd} Place Graduate Team

• \emph{Weed Science Society of America Meeting}, San Juan · PR
\hfill February 2015\\
1\textsuperscript{st} Place Outstanding Poster Presentation (Weed
Biology section)

• \emph{Western Weed Science Society Weed Contest}, Columbus · OH · USA
\hfill July 2015\\
3\textsuperscript{rd} Place Graduate Team and 3\textsuperscript{rd}
Place Overall Individual Graduate

• \emph{Mayrink Vieira High School Science Fair}, Ipatinga · MG · BRA
\hfill September 2002\\
Outstanding team project ``Batteries and electrolysis''

• \emph{Mayrink Vieira High School Science Fair}, Ipatinga · MG · BRA
\hfill September 2001\\
Outstanding team project ``Cell organelles''

\newpage

\hypertarget{professional-society-memberships}{%
\section{PROFESSIONAL SOCIETY
MEMBERSHIPS}\label{professional-society-memberships}}

• Weed Science Society of America

• North Central Weed Science Society

\hypertarget{manuscripts-in-preparation}{%
\section{MANUSCRIPTS IN PREPARATION}\label{manuscripts-in-preparation}}

• \textbf{Oliveira MC}, Butts T, Conley S, and Werle R. Tolerance of US
Midwest Soybean Cultivars to Preemergent Applications of Metribuzin and
Sulfentrazone. Target Journal: \emph{Weed Technology}

• \textbf{Oliveira MC}, Giacomini D, Tranel P, and Werle R. Molecular
and greenhouse validation of field evolved resistance to glyphosate and
PPO-inhibitors in Nebraska. Target Journal: \emph{Pest Management
Science}

• Werle R, \textbf{Oliveira MC}, and Butts L (2019) Combining herbicide
programs and cereal rye cover crop for integrated weed management in
soybeans. Target Journal: \emph{Agriculture}, invited submission for
special issue entitled: ``Cover Crops''

• Smith D, \textbf{Oliveira MC}, Davis V, and Werle R. (2019) Herbicide
carryover evaluation following commonly applied silage corn and soybean
herbicides. Target Journal: \emph{Agronomy}

• Soltani N, \textbf{Oliveira MC}, Alves G, Werle R, Kruger G,
Norsworthy J, Sprague C, Young B, Reynolds D, and Sikkema P (2019)
Large-Scale Off-Target Movement Assessment of Dicamba in North America.
Target Journal: \emph{Weed Technology}

• Oliveira MC, Ospitan OA, Begcy K, Werle R (2019) Preventing weed
establishment while promoting increased sustainability of agricultural
systems: win-win! Target Journal: \emph{Plant Science} (Invited)

\hypertarget{submitted-manuscript}{%
\section{SUBMITTED MANUSCRIPT}\label{submitted-manuscript}}

• \textbf{Oliveira MC}, da Silva AL, Ulguim AR, Werle R. Assessment of
Weed Management Strategies and Challenges in Brazilian Cropping Systems.
Submitted to \emph{Weed Technology}

\hypertarget{publications}{%
\section{PUBLICATIONS}\label{publications}}

• Soltani S, \textbf{Oliveira MC}, Alves GS, Werle R, Norsworthy JK,
Sprague CL, Young BG, Reynolds DB, Brown A, and Sikkema PH (2019)
Off-Target Movement Assessment of Dicamba in North America. in press
\emph{Weed Technology}. doi:
\href{https://doi.org/10.1017/wet.2020.17}{10.1017/wet.2020.17}

• \textbf{Oliveira MC}, Butts L, and Werle R (2019) Survey of cover crop
management in Nebraska. 9:1-14 \emph{Agriculture}, invited submission
for special issue entitled: ``Cover Crops''. doi:
\href{https://doi.org/10.3390/agriculture9060124}{10.3390/agriculture9060124}

• Knezevic SZ, Pavlovic P, Barnes ER, Beiermann C, \textbf{Oliveira MC},
Lawrence N, Scott JE, and Jhala AJ (2019) Critical Time for Weed Removal
in Glyphosate-Resistant Soybean as Influenced by Preemergence
Herbicides. 33-393:399. \emph{Weed Technology}. doi:
\href{https://doi.org/10.1017/wet.2019.18}{10.1017/wet.2019.18}

• Ferreira EA, Paiva MCG, Pereira GAM, \textbf{Oliveira MC}, Silva, EB
(2019) Fitossociologia de plantas daninhas na cultura do milho submetida
à aplicação de doses de nitrogênio, \emph{Journal of Neotropical
Agriculture}, 6:100-107 (in Portuguese with English abstract). doi:
\href{https://doi.org/10.32404/rean.v6i2.2710}{10.32404/rean.v6i2.2710}

• Werle R, \textbf{Oliveira MC}, Jhala AJ, Klein R, Proctor CA, and Rees
J (2018) Survey of Nebraska farmers' adoption of dicamba-tolerant
soybean technology and off-target movement. \emph{Weed Technology},
32:754--761. doi:
\href{https://doi.org/10.1017/wet.2018.62}{10.1017/wet.2018.62}

• \textbf{Oliveira MC}, Gaines TA, Patterson EP, Jhala AJ, Irmak S,
Amundsen K, and Knezevic SZ (2018) Interspecific and intraspecific
transfer of metabolism based mesotrione resistance transfer in dioecious
weedy \emph{Amaranthus}. \emph{Plant Journal}. 96:1051--1063. doi:
\href{https://onlinelibrary.wiley.com/doi/full/10.1111/tpj.14089}{10.1111/tpj.14089}

• Knezevic SZ, Ospitan AO, \textbf{Oliveira MC}, and Scott JE (2018)
\emph{Lythrum salicaria} (purple loosestrife) control with herbicides:
multiyear applications. \emph{Invasive Plant Science and Management}.
11:143--154. doi:
\href{https://doi.org/10.1017/inp.2018.17}{10.1017/inp.2018.17}

• \textbf{Oliveira MC}, Pereira GAM, Ferreira EA, Barbosa JB, Knezevic
SZ, and Werle R (2018) Additive design: the concept and data analysis.
\emph{Weed Research}, 58:338--347. doi:
\href{https://onlinelibrary.wiley.com/doi/full/10.1111/wre.12317}{10.1111/wre.12317}

• \textbf{Oliveira MC}, Gaines TA, Jhala AJ, and Knezevic SZ (2018)
Inheritance of mesotrione resistance in an (\emph{Amaranthus
tuberculatus}) (var. \emph{rudis}) population from Nebraska, USA.
\emph{Frontiers in Plant Science}, 9:60. doi:
\href{https://www.frontiersin.org/articles/10.3389/fpls.2018.00060/full}{10.3389/fpls.2018.00060}

• Miller JJ, Schepers JS, Shapiro CA, Arnenson NJ, Eskridge KM,
\textbf{Oliveira MC}, and Giesler LJ (2018) Characterizing soybean vigor
and productivity using multiple crop canopy sensor readings. \emph{Field
Crops Research}, 216:22--31. doi:
\href{https://www.sciencedirect.com/science/article/pii/S0378429017310341}{10.1016/j.fcr.2017.11.006}

• \textbf{Oliveira MC}, Gaines TA, Dayan FE, Patterson EL, Jhala AJ, and
Knezevic SZ (2018) Reversing resistance to tembotrione in an
\emph{Amaranthus tuberculatus} (syn. \emph{rudis}) population from
Nebraska, USA with cytochrome P450 inhibitors. \emph{Pest Management
Science}, 74:2296--2305. (Invited publication) doi:
\href{https://onlinelibrary.wiley.com/doi/full/10.1002/ps.4697}{10.1002/ps.4697}

• \textbf{Oliveira MC}, Feist D, Eskelsen S, Scott JE, and Knezevic SZ
(2017) Weed control in soybean with preemergence and
postemergence-applied herbicides. \emph{Crop, Forage, and Turfgrass
Management}, 3:1--7

• \textbf{Oliveira MC}, Jhala A, Gaines T, Irmak S, Amundsen K, Scott
JE, and Knezevic SZ (2017) Confirmation and Control of HPPD-inhibiting
Herbicide-Resistant Waterhemp (\emph{Amaranthus tuberculatus}) in
Nebraska. \emph{Weed Technology}, 31:67--79. doi:
\href{https://www.cambridge.org/core/journals/weed-technology/article/confirmation-and-control-of-hppdinhibiting-herbicideresistant-waterhemp-amaranthus-tuberculatus-in-nebraska/69C31C7039DBE3FD49A55C73EEE5F2EE}{10.1017/wet.2016.4}

• Butts TR, Miller JJ, Pruitt JD, Vieira BC, \textbf{Oliveira MC},
Ramirez II S, and Lindquist JL (2017) Light quality effect on corn
growth as influenced by weed species and nitrogen rate. \emph{Journal of
Agricultural Science}, 9:15--27. doi:
\href{http://www.ccsenet.org/journal/index.php/jas/article/view/63437}{10.5539/jas.v9n1p15}

• Braga RR, Santos JB, Zanuncio JC, Bibiano CS, Ferreira EA,
\textbf{Oliveira MC}, Silva DV, and Serrāo JE (2016) Effect of growing
\emph{Brachiaria brizantha} on phytoremediation of picloram under
different pH environments. \emph{Ecological Engineering}, 94:102--106.
doi:
\href{https://www.sciencedirect.com/science/article/pii/S0925857416302853}{10.1016/j.ecoleng.2016.05.050}

• Magalhães JES, Ferreira EA, \textbf{Oliveira MC}, Pereira GAM, Silva
DV, and Santos JB (2016) Effect of plant-biostimulant on cassava initial
growth. \emph{Ceres}, 63:208--213 doi:
\href{http://www.scielo.br/scielo.php?script=sci_arttext\&pid=S0034-737X2016000200208}{10.1590/0034-737X201663020012}

• Silva DV, Ferreira, \textbf{Oliveira MC}, Pereira GAM, Braga RR,
Santos JB, Aspiazú I, and Souza MF (2016) Productivity of cassava and
other crops in an intercropping system. \emph{Ciencia e Investigación
Agraria}, 43:159--166. doi:
\href{https://scielo.conicyt.cl/scielo.php?script=sci_arttext\&pid=S0718-16202016000100015}{10.4067/S0718-16202016000100015}

• Pereira GAM, \textbf{Oliveira MC}, Braga RR, Silva DV, Oliveira AJM,
Fernandes JSC, and Andrade Junior VC (2015) Growth of carrot cultivars
in different environments. \emph{Comunicata Scientiae}, 6:317--325. (in
Portuguese with English abstract)

• Santos RC, Ferreira EA, Santos JB, \textbf{Oliveira MC}, Silva DV,
Pereira GAM, Galon L, Aspiazú I, and Mattos NP (2015) Phytosociological
characterization of weed species as affected by soil management.
\emph{Australian Journal of Crop Science}, 9:112--119

• Pereira GAM, \textbf{Oliveira MC}, Oliveira, AJM, Fernandes JSC,
Andrade Junior VC, Silva DV, and Ferreira EA (2015) Performance of
carrot genotypes at two Jequitinhonha Valley sites. \emph{Semina:
Ciências Agrárias}, 36:4059-4070. doi:
\href{http://www.uel.br/revistas/uel/index.php/semagrarias/article/view/17819}{10.5433/1679-0359.2015v36n6Supl2p4059}

• Ferreira EA, Silva DV, Braga RR, \textbf{Oliveira MC}, Pereira GAM,
Santos JB, and Sediyama T (2014) Initial growth of cassava in
intercropping system. \emph{Scientia Agraria Paranaensis}, 13:219-226

• Franco MHR, Nery MC, França AC, \textbf{Oliveira MC}, Franco GCN, and
Lemos VT (2013) Production and physiological quality of bean seeds after
application diquat. \emph{Semina: Ciências Agrárias}, 34:1707-1714 (in
Portuguese with English abstract). doi:
\href{http://www.uel.br/revistas/uel/index.php/semagrarias/article/view/10234}{10.5433/1679-0359.2013v34n4p1707}

• Pereira GAM, Lemos VT, Santos JB, Ferreira EA, Silva DV,
\textbf{Oliveira MC}, and Menezes CWG (2012) Growth of cassava and weed
in response to phosphate fertilizer. \emph{Ceres}, 59:716- 722 (in
Portuguese with English abstract). doi:
\href{http://www.scielo.br/scielo.php?script=sci_arttext\&pid=S0034-737X2012000500019}{10.1590/S0034-737X2012000500019}

• Miranda VS, Ribeiro KG, Silva AC, Pereira RC, Pereira OG, Torrado PV,
Fernandes JSC, and \textbf{Oliveira MC} (2012) Rehabilitation with
forage grasses of an area degraded by urban solid waste deposits.
\emph{Brazilian Journal of Animal Science}, 41:18-23. doi:
\href{http://www.scielo.br/scielo.php?script=sci_arttext\&pid=S1516-35982012000100003}{10.1590/S1516-35982012000100003}

\hypertarget{invited-presentations}{%
\section{INVITED PRESENTATIONS}\label{invited-presentations}}

• \emph{Valoriza Agronegocios}, Patos de Minas · MG · Brazil
\hfill January 2019\\
Lecture: ``Treinamento resistência, dicamba, identificação de plantas
daninhas e herbicidas'' (In Portuguese)

• \emph{II IMAST}, Dracena · SP · Brazil \hfill September 2018\\
Lecture: ``Da resistência ao dicamba: o desafio no manejo de plantas
daninhas nos EUA'' (In Portuguese)

• \emph{JF Crow Institute for the Study of Evolution}, Madison · WI ·
USA \hfill September 2018\\
Lecture: Evolution of 4-hydroxyphepylpiruvate dioxygenase inhibitor
herbicide resistance in \emph{Amaranthus tuberculatus}

\hypertarget{extension-presentations}{%
\section{EXTENSION PRESENTATIONS}\label{extension-presentations}}

• Waterhemp Resistance Evolution and Status. Waterhemp worshop, Chippewa
Falls, Bangor, Janesville, Appleton, WI, USA, March 2019

• Weed control in corn with different PRE-applied herbicides and carrier
volumes. Wisconsin Cropping Systems Weed Science Field Day, Arlington,
WI, USA, July 2018

• Integrated weed management strategies in cassava fields. Field day of
cassava: specificities and practices, Diamantina, MG, Brazil, October
2011

\hypertarget{events-organizations}{%
\section{EVENTS ORGANIZATIONS}\label{events-organizations}}

• Wisconsin Weed Science Field Day, Arlington and Lancaster · WI · USA
\hfill July 2018

• Flaming Weed Control Workshop, Concord · NE · USA \hfill August
2014/15/16/17

• Field day of cassava: specificities and practices, Diamantina, MG,
Brazil \hfill October 2011

• II Seminar of Forage and Rangeland Sciences, Diamantina · MG · Brazil
\hfill March 2009

• I Seminar of Engineers, Diamantina · MG · Brazil \hfill September 2009

\hypertarget{professional-meeting-presentation-and-abstracts}{%
\section{PROFESSIONAL MEETING PRESENTATION AND
ABSTRACTS}\label{professional-meeting-presentation-and-abstracts}}

• Werle R, DeWerff R, Striegel S, Arsenijevic N, Ribeiro VHV,
\textbf{Oliveira MC} (2019) Systems Approach to Weed Management in Corn
in Wisconsin. Proceeding of the Weed Science Society of America Annual
Meetings, New Orleans, LA

• \textbf{Oliveira MC}, Jhala A, Proctor C, Mitchell P, Werle R (2019)
Survey of 2017 and 2018 dicamba use in Nebraska and Wisconsin soybean
production systems. Proceeding of the Weed Science Society of America
Annual Meetings, New Orleans, LA

• \textbf{Oliveira MC}, Bernards M, Jhala A, Proctor C, Stepanovic S,
Werle R (2019) Adaptation of Palmer Amaranth to the Upper Midwest.
Proceeding of the Weed Science Society of America Annual Meetings, New
Orleans, LA

• \textbf{Oliveira MC}, Lencina A, Ulguim A, Werle R (2019) A
cropping-system weed science survey of Brazil, a breadbasket country in
the tropics. Proceeding of the Weed Science Society of America Annual
Meetings, New Orleans, LA

• \textbf{Oliveira MC}, Giacomini D, Tranel D, Vieira GS, Arsenijevic N,
Werle R (2018) Molecular and greenhouse validation of field-evolved
resistance to glyphosate and PPO-inhibitors in Palmer amaranth.
Proceeding of the North Central Weed Science Society Annual Meetings,
Milwaukee, WI

• Butts TR, \textbf{Oliveira MC}, Arsenijevic MC, Conley S, Werle R
(2018) Tolerance of midwest soybean cultivars to preemergent
applications of metribuzin and sulfentrazone (2018). Proceeding of the
North Central Weed Science Society Annual Meetings, Milwaukee, WI

• Liu L, \textbf{Oliveira MC}, Werle R (2018) Wisconsin cropping systems
weed science survey - where are we at? Proceeding of the North Central
Weed Science Society Annual Meetings, Milwaukee, WI

• Renz R, Striegel S, DeWerff R, Arsenijevic N, Ribeiro VHV,
\textbf{Oliveira MC}, Luck B, Werle R (2018) Impact of carrier volume
rate on PRE-emergence herbicide efficacy in Wisconsin cropping systems.
Proceeding of the North Central Weed Science Society Annual Meetings,
Milwaukee, WI

• Ribeiro VHV, \textbf{Oliveira MC}, Smith D, Santos JB, Werle R (2018)
Spectrum of Weed Species Controlled by Various PRE- Emergence Soybean
Herbicides in Wisconsin. Proceeding of the North Central Weed Science
Society Annual Meetings, Milwaukee, WI

• Arsenijevic N, Striegel S, Ribeiro VHV, \textbf{Oliveira MC}, Werle R
(2018) Screening of soybean variety tolerance to PRE-emergence
herbicides sulfentrazone (PPO) and metribuzin (PSII). Proceeding of the
North Central Weed Science Society Annual Meetings, Milwaukee, WI

• \textbf{Oliveira MC}, Patterson E, Gaines TA, Knezevic SZ (2017) A
qualitative assay for detecting Palmer amaranth and its hybrids amongst
pigweed species. Proceeding of the North Central Weed Science Society
Annual Meetings, Saint Louis, MO

• \textbf{Oliveira MC}, Gaines TA, Knezevic SZ (2017) Inheritance of
mesotrione resistance in a waterhemp population from Nebraska.
Proceeding of the North Central Weed Science Society Annual Meetings,
Saint Louis, MO

• Vieira G, Oliveira MC, Giacomini D, Arsenijevic N, Tranel P, Werle R
(2017) Molecular screening of PPO and glyphosate resistance in Palmer
amaranth populations from Nebraska. Proceeding of the North Central Weed
Science Society Annual Meetings, Saint Louis, MO

• Miller JJ, Schepers JS, Giesler LJ, Shapiro CA, Oliveira MC, Arnenson
NJ (2016) Characterizing soybean vigor and productivity using multiple
crop canopy sensor readings. Proceeding of the American Society of
Agronomy Annual Meetings, Phoenix, AZ

• \textbf{Oliveira MC}, Dayan FE, Gaines TA, Knezevic SZ (2016)
Investigations into target-site and nontarget site resistance mechanisms
in HPPD resistant waterhemp from Nebraska. Proceeding of the North
Central Weed Science Society Annual Meetings, Des Moines, IA

• \textbf{Oliveira MC}, Scott JE, Knezevic SZ (2016) Critical time for
weed removal in soybeans is influenced by soil applied herbicides.
Proceeding of the North Central Weed Science Society Annual Meetings,
Des Moines, IA

• Knezevic SZ, Oliveira MC, Scott JE (2016) Simulated drift of dicamba
on potentially sensitive crops. Proceeding of the North Central Weed
Science Society Annual Meetings, Des Moines, IA

• \textbf{Oliveira MC}, cott JE, Jhala AJ, Gaines T, Knezevic SZ (2016)
Herbicide Programs to control HPPD-resistant Common Waterhemp in
Nebraska. Proceeding of the Weed Science Society of America Annual
Meetings, San Juan, PR

• \textbf{Oliveira MC}, Gaines T, Jhala A, Knezevic SZ (2016)
Pollen-mediated resistance transfer from HPPD-resistance waterhemp to
susceptible Palmer amaranth in Nebraska. Proceeding of the Weed Science
Society of America Annual Meetings, San Juan, PR

• \textbf{Oliveira MC}, Oliveira e Freitas IF, Scott JE, Knezevic SZ
(2015) Soybean yields and critical time for weed removal as influenced
by soil applied herbicides. Proceeding of the North Central Weed Science
Society Annual Meetings, Indianapolis, IN

• \textbf{Oliveira MC}, Gaines T, Jhala A, Knezevic SZ (2015)
HPPD-resistance in waterhemp from Nebraska: Evidence for multiple genes
enhancing metabolism. Proceeding of the North Central Weed Science
Society Annual Meetings, Indianapolis, IN

• \textbf{Oliveira MC}, Gaines T, Knezevic SZ (2014) HPPD Resistance in
Nebraska's Waterhemp: Enhance Metabolism of Target Site Mutation.
Proceeding of the North Central Weed Science Society Annual Meetings,
Minneapolis, MN

• \textbf{Oliveira MC}, Gaines T, Knezevic SZ (2014) Herbicide Options
to Control HPPD resistant waterhemp in Nebraska. Proceeding of the North
Central Weed Science Society Annual Meetings, Minneapolis, MN

• \textbf{Oliveira MC}, Scott JE, Franssen AS, Shivrain VK, Knezevic SZ
(2014) Confirmation of HPPD Resistant Common Waterhemp in Nebraska.
Proceeding of the North Central Weed Science Society Annual Meetings,
Minneapolis, MN

• Knezevic SZ, Jhala AJ, Kruger GR, Sandell LD, \textbf{Oliveira MC},
Golus JA, Scott J (2014) Volunteer Roundup Ready Soybean Control in
Roundup Ready Corn. Proceeding of the North Central Weed Science Society
of America Annual Meetings, Minneapolis, MN

• \emph{30+ abstracts in Brazilian Professional meetings from 2008 to
2013}

\hypertarget{grants}{%
\section{GRANTS}\label{grants}}

• \href{https://oeo.cals.wisc.edu}{UW-Madison Division of Extension and
UW-Madison CALS} \hfill March 2019\\
Adaptation of Industrial Hemp (\emph{Cannabis sativa}) to Agronomic
practices in Wisconsin Funded: \$35,000

• \href{http://wicorn.org}{Wisconsin Corn Promotion Board}
\hfill January 2019\\
Title: Challenging the Weed Management Challenges in Wisconsin Corn
Production Funded: \$30,000

• \href{http://www.ncsrp.com/}{North Central Soybean Research Program}
\hfill August 2018\\
Title: Adaptation and Management of Glyphosate-Resistant Pigweed Species
in the Northern Midwest US Soybean Production Systems\\
Submitted but not funded

• \href{http://www.capes.gov.br/}{CAPES} Foundation \hfill September
2013\\
Title: Evolution of HPPD-resistant waterhemp (\emph{Amaranthus
tuberculatus}) in Nebraska\\
Funded my Ph.D at the University of Nebraska-Lincoln (\$130,000)

\hypertarget{extension-manuals}{%
\section{EXTENSION MANUALS}\label{extension-manuals}}

• Cover Crops 101 \hfill August 2019\\
\href{https://learningstore.extension.wisc.edu/collections/farming/products/cover-crops-101}{Link}

• Herbicide site of action key for crop injury symptoms \hfill September
2018\\
\href{http://www.wiscweeds.info/img/2018\%20Herbicide\%20SOA\%20injury\%20chart/2018_HerbicideInjury_EN.pdf}{English}
and
\href{https://maxweeds.rbind.io/files/Herbicide\%20chart/2018_HerbicideInjury_BRA.pdf}{Portuguese}
version.

• UW Waterhemp Challenge: Preliminary Report Comparison of Soil Residual
Herbicides \hfill August 2018\\
\href{http://www.wiscweeds.info/img/2018\%20waterhemp\%20challenge/Waterhemp\%20Management\%20PRE\%20Comparison_Lancaster\%20WI\%202018.pdf}{Link}

\hypertarget{programming}{%
\section{PROGRAMMING}\label{programming}}

• \href{https://www.r-project.org}{R} e
\href{https://www.rstudio.com}{Rstudio}

• \href{https://github.com}{GitHub}

• \href{https://latex.org}{LaTeX} e
\href{https://www.markdownguide.org}{Markdown}

• Agricultural Research Management
\href{https://www.arm.com/products/development-tools}{(ARM)}

• Statistical Analysis Software \href{https://www.sas.com/}{(SAS)}

\hypertarget{personal-projects}{%
\section{PERSONAL PROJECTS}\label{personal-projects}}

• Founder of Open Weed Science Community:
\url{https://www.openweedsci.org} -- A website with tutorial for codes
of data analysis and data visualization commonly published in
manuscripts of weed research and related areas

• Blog: \url{https://maxweeds.rbind.io} - My blog developed with Hugo,
blogdown, Rstudio, and GitHub

\hypertarget{languagues}{%
\section{LANGUAGUES}\label{languagues}}

• Portuguese (Native speaker proficiency)

• English (Bilingual proficiency)

• Spanish (Intermediate working proficiency)

\end{document}